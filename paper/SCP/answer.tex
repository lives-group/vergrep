\documentclass{article}

%\newcommand{\tid}{\mbox{\texttt{T}}_{id}}
%\newcommand{\simd}{$\mu$-\textsc{Simd}}

\usepackage{ucs}
\usepackage[utf8x]{inputenc}
\usepackage{graphicx}
\usepackage{amsmath,amsthm}
\usepackage{amssymb}
\usepackage{url}
\usepackage{stmaryrd}
\usepackage{ifpdf, proof}

%\newcommand\pgcc{\mbox{\textsf{pgcc}}}
%\newcommand\ppgc{\mbox{\textsf{ppgc}}}
%\newcommand\accull{\mbox{\textsf{AccUll}}}
%\newcommand\psyche{\mbox{\textsf{PsycheC}}}

%\newcommand{\muA}{$\mu{A}$}
%\newcommand{\muB}{$\mu{B}$}
%\newcommand{\muC}{$\mu{C}$}
%\newcommand{\muD}{$\mu{D}$}

%\newcommand\doubleplus{+\kern-1.3ex+\kern0.8ex}

%\def\ppcg{\textsf{ppcg}}
%\def\accull{\textsf{AccULL}}
%\def\etino{\textsf{Etino}}
%\def\dawncc{\textsf{dawncc}}

\date{Ouro Preto, \today}

%\include{macros}

\begin{document}

\noindent
Dear Editor,

\vspace{1cm}

we discuss below all reviewer's comments on our paper Certified
Derivative Based Parsing of Regular Expressions. We have included some
examples and fixed some errors, to address comments by the reviewers,
to which we thank for careful reading that has allowed a significant
improvement of the paper.

Sincerely,

\vspace{1cm}

%\begin{flushright}
\noindent
Sincerely,
%\end{flushright}

% \vspace{0.6cm}

%\begin{flushright}
\vspace{1cm}

\noindent
Rodrigo Ribeiro, Raul Lopes e Carlos Camar\~ao.
%\end{flushright}

\vspace{1cm}

%%%%%%%%%%%%%%%%%%%%%%%%%%%%%%%%%%%%%%%%%%%%%%%%%%%%%%%%%%%%%%%%%%%%%%%%%%%%%%%%


\section*{Reviewer 1}

\begin{itemize}

    \item p2, l43 Authors claim "A formalization of Brzozowski derivatives based RE parsing in Agda"

          The authors formalize some aspects of derivatives in Agda. But I couldn't find
          a *clearly specified* and certified parsing algorithm in the paper.
          The material presented on pages 17 and 18 seems connected to the claim but
          where's the parsing algorithm including its correctness proof?

    \textsc{Answer}: We understand that the parsing problem for RE consists in finding all matching prefixes and 
          substrings for a given input string and some RE e. In section 4.4, we present types that specify 
          when an input string has a prefix / substring that matches some RE and decidability proofs 
          for such types. Under the propositions-as-types interpretation such proofs are certified algorithms 
          for finding prefixes and substrings for a given RE.

     \item p2, l50 Authors claim "A detailed explanation of the technique used to quotient derivatives with respect to ACUI axioms"
           This is not true! The authors implement a few (simple) identities such as epsilon . r = r\\
           ACI is not dealt with at all. It's an essential step to guarantee that the number of dissimilar derivatives (under ACI) is finite.
           
    \textsc{Answer}: You are right! We removed all references to ACUI axioms and mention explicitly that we use smart constructors to 
           simplify RE using some well known equivalences.       
          

    \item p3, l13 Authors claim "A formalization of Antimirov's derivatives ..."
          The authors use lists to represent the set of *partial* derivatives. Yet again, I(demptotence) is simply dropped.
          Antimirov uses a smart constructor to simplify epsilon . r to r, it's only a minor detail, but again
          something which the authors didn't discuss.
          
    \textsc{Answer}: We didn't mention Antimirov simplification step because we are already using it for Brzozowski derivatives.
        Also, such simplification step doesn't affect correctness of the Antimirov partial derivative based parsing
        algorithm.
        
        
    \item The authors report poor performance result. This is no surprise as the size of derivatives and partial derivatives
          will explode because the authors formalization doesn't simplify derivatives based on ACI
          and the list representation used for partial derivatives contains duplicates.
    
    \textsc{Answer}: The number of derivatives is only relevant when one is building a finite state machine from a given RE.
    Quotienting resulting REs under ACUI guarantees the finite number of dissimilar derivatives ensuring termination
    of the automata construction. For parsing, simplyfing under all axioms of ACUI isn't necessary since the number
    of dissimilar derivatives is irrelevant, because termination of RE derivative based parsing depends on the 
    size of the input string not on the number of derivatives of a given RE.      

    \item p3, l13     Antimorov's *partial* derivatives
    
    \textsc{Answer}: Fixed.
    
    \item p8, l14    "...inversion lemmas...", at least provide some details, otherwise, the reader is left clueless
    
    \textsc{Answer:} We have added the inversion lemma for choice and single character RE and some text explaining these definitions.
    
    \item agda instead of Agda, various places
 
    \textsc{Answer}: Fixed. 
    
    \item The contributions are not significant enough. In fact, the contributions claimed
          at the beginning of the paper are not substantiated at all.
          
    \textsc{Answer}: We appreciate Reviewer 1's detailed comments which allowed us to fix and improve the presentation of our work. 
     In our view the text now addresses Reviewer 1's concerns:
     \begin{enumerate}
         \item References to ACUI axioms were removed, as they are not needed for
RE derivative based parsing. The text now explicitly mentions that
smart constructors only perform simnplifications based on RE
equivalences.
         \item We have added some examples of: RE semantics relation, Brzozowski derivatives, Antimirov partial derivatives 
               and parsing using derivatives. 
     \end{enumerate}
     
     \item The structure can be improved (mainly a boring sequences of Agda formalization and results).
           I couldn't find a single worked out example!
           
     \textsc{Answer}: We have added some examples in the paper: RE semantics relation, Brzozowski derivatives,
     Antimirov partial derivatives and derivative based parsing.
\end{itemize}

\section*{Reviewer 2}

\begin{itemize}
    \item The paper is clear and in general well-written. 
    
    \textsc{Answer}: Thank you!
    
    \item l. 33: until his work $\to$ until their work (or it is his work because there you are referring to Brzozowski?)
    
    \textsc{Answer:} Fixed.
    
    \item l. 34: interest on its use $\to$ interest on their use

    \textsc{Answer}: Fixed.
    
    \item l. 39: that has been developed by us, using $\to$ we developed by using

    \textsc{Answer}: Fixed.

    \item l. 13: derivatives and its use $\to$ derivatives and their use

    \textsc{Answer}: Fixed.

    \item l. 14: of types of types $\to$ of types

    \textsc{Answer}: Fixed.

    \item l. 42: Dec p $\to$ Dec P

    \textsc{Answer}: Fixed.

    \item l. 36: suc $\to$ S

    \textsc{Answer}: Fixed.

    \item l. 12: In our Agda formalization, *in contrast*, we represent....

    \textsc{Answer}: Fixed.

    \item l. 31: Bigger $\to$ Composite

    \textsc{Answer}: Fixed.

    \item l. 36: constructor*s* specifies

    \textsc{Answer}: Fixed.

    \item In the definition of RE semantics:
    perhaps  mention that in Agda one can overload constructor names. 

    \textsc{Answer}: We add the following sentences before the definition of RE semantics data type:
    "Agda allows the overloading of constructor names. In some cases we use the same symbol in 
    the RE syntax and in the definition of its semantics."

    \item l. 42: \$ c $\to$ [c] (outside the double brackets)

    \textsc{Answer}: Fixed.

    \item l. 43: Is it neccesary to define zs? couldn't you directly say xs++ys $\in$ [[l $\bullet$ r]], instead of zs = xs++ys $\to$ zs $\in$ [[l $\bullet$ r]]?

    \textsc{Answer}: It is possible to define the semantics of concatenation without mentioning zs. We introduce this extra equality because it simplifies
    some proofs.  

    \item l. 53: Simply for uniformity, perhaps you can speak about character c instead of a in order to use the same name as in the data definition.

    \textsc{Answer}: Fixed.

    \item l. 9: from a parse tree from l (r) $\to$ from a parse tree for l (r)

    \textsc{Answer}: Fixed.

    \item In section 4, perhaps it is useful if you present examples of derivatives

    \textsc{Answer}: We added examples of derivatives and partial derivatives.

    \item l. 47: is proved by function $\nu[e]$ $\to$ is proved by function $\nu[\_]$

    \textsc{Answer}: Fixed.

    \item l. 29: RE's $\to$ REs

    \textsc{Answer}: Fixed.

    \item l. 35: [8], we use smart $\to$ [8]. We use smart

    \textsc{Answer}: Fixed.

    \item Mention somewhere that smart constructors are denoted by prefixing a back quote to the constructor name.

    \textsc{Answer}: The following sentence was introduced before the definitions of the smart constructors:
    "As a convention, a smart constructors is named by prefixing the constructor name with a back quote."


    \item Instead of stating 6 different lemmas, I would join the soundness lemmas in only 
     one with cases as 1, 2 and 3 (and a proof for each case), and the same with the completeness lemmas.

    \textsc{Answer}:  We have simplified the presentation by combining soundness and completeness lemmas for each smart constructor.

    \item l. 32: one of the REs are $\to$ one of the REs is

    \textsc{Answer}: Fixed.

    \item l. 17: a RE with respect $\to$ a RE e with respect

    \textsc{Answer}: Fixed.

    \item l. 40: (section title) and its Properties $\to$ and their Properties

    \textsc{Answer}: Fixed.

    \item l. 48: collectively accepts $\to$ collectively accept

    \textsc{Answer}: Fixed.

    \item l. 50: ACUI equations holds $\to$ ACUI equations hold
    
    \textsc{Answer}: Fixed.

    \item l. 30: agda $\to$ Agda

    \textsc{Answer}: Fixed.

    \item There is a mismatch in the order of the parameters between Agda's definition of $\odot$ and the formal definition of $\odot$ in line 28 (S $\odot$ e').

    \textsc{Answer}: Fixed.

    \item l. 21: theorems $\to$ theorem

    \textsc{Answer}: Fixed.

    \item l. 50: encode $\to$ encodes

    \textsc{Answer}: Fixed.

    \item From the explanation on page 16 about what "matching a prefix" and "matching a substring" mean one would expect that 
    the definitions of those concepts are given by sigma types. Perhaps it would be helpful to mention that the GADTs 
    provided in fact express such existentials.

    \textsc{Answer}: A sentence was added explaining that such predicates (prefix and substring) could be easily defined by dependent products.
    New types were defined for clarity.

    \item In the definition of constructor String of IsSubstring a ($\forall$ (ys zs ws)) is missed.

    \textsc{Answer}: We omit such parameters to avoid clutter. In Section 2, we have:
    "To avoid clutter, we'll omit implicit arguments from the source code presentation. The reader can safely assume that every free variable in a type is an implicity parameter."

    \item l. 9: better refer to Lemma 13

    \textsc{Answer}: Fixed.

    \item l. 31: e $\to$ xs

    \textsc{Answer}: Fixed.


    \item l. 54: 'be' appears twice
          l. 54: the Brzozowski definition $\to$ Brzozowski's definition

    \textsc{Answer}: Fixed.

    \item l. 50: drop a '.'

    \textsc{Answer}: Fixed.

    \item l. 57: \verb|\cite| missed (Norell2009)

    \textsc{Answer}: Fixed.

    \item l. 9: a lot $\to$ much

    \textsc{Answer}: Fixed.

    \item l. 22: CFG $\to$ a CFG

    \textsc{Answer}: Fixed.

    \item l. 45: shows $\to$ show

    \textsc{Answer}: Fixed.

    \item l. 11: have given $\to$ gave

    \textsc{Answer}: Fixed.

    \item A section name ("References") needs to be added.

    \textsc{Answer}: Fixed.

    \item References [5] and [25] are the same.

    \textsc{Answer}: Fixed.

    \item Reference [12] is the SBLP 2016 paper, isn't it? Year is 2016 and should say SBLP instead of Programming Languages.

    \textsc{Answer}: We cited the book chapter on LNCS 9889 which consists of the proceedings for SBLP 2016. 
        
\end{itemize}

\section*{Reviewer 3}

\begin{itemize}
    
    \item For example, I had to go look up "indexed datatypes" in an Agda tutorial
    to understand the N $\to$ Set and A $\to$ Set parts in the data definitions of Vec
    and equality on pages 4 and 5. It seems that the text has been kept from
    the previous version of the paper, that used Idris for the mechanization,
    but the notation that Idris use is more explicit. In the definition of Dec,
    shouldn't the lowercase p in the type of "no" be an uppercase P?

    \textsc{Answer}: Yes! We fixed the lower case p in the definition of Dec type.
    We also add text to explain the concepts of indexed and parameterised types in
    Section 2.

    \item The section on "smart constructors" could be greatly shortened. They are
    not essential to the correctness of the derivative parsing algorithm,
    just for performance, and the identities and accompanying lemmas are
    very simple.

    \textsc{Answer}: We joined the soundness and completeness lemmas for each smart constructor to
    reduce the description of such functions.

    \item On the other hand, the explanation about Brzozowski derivatives could
    be expanded with a concrete example of why they are interesting for
    regular expression parsing.

    \textsc{Answer}: We modify the sections on Brzozowski derivatives, Antimirov partial derivatives 
    and parsing by adding examples and explaining how to use derivatives for parsing.

    \item The definition of Antimirov's derivatives is using the symbol for
    the Brzozowski derivative in the right side of concatenation.

    \textsc{Answer}: Yes, fixed!

    \item My recommendation is to accept after minor revisions, basically to
    better explain both the use of Agda and an informal explanation of why
    derivatives of regular expressions work and why they are interesting
    for parsing. The experimental results also need better analysis, as
    I am not convinced that the two reasons they gave tell the whole story;
    the other implementations should be better by a constant factor, but
    with the exception of haskell-regexp they are exhibiting better asymptotic
    behavior.

    \textsc{Answer}: Thank you for your careful reading. We extended the text in order
    to better explain (partial) derivatives and their use
    for parsing and examples of these concepts. We also add some explanation about
    indexed and parameterized data types in Section 2. The reason of why tools like 
    GNU-Grep and Google re2 are exhibiting better asymptotic behaviour needs further 
    investigation on how such tools are implemented and how efficient is the code 
    generated by Agda's compiler. 
\end{itemize}    

\end{document}


